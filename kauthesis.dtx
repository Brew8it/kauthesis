% \iffalse meta-comment
%% kauthesis.dtx
%% Copyright (c) 2011-2012 Stefan Berthold <stefan.berthold@kau.se>
%
% This file is part of the kauthesis bundle.
%
% This work may be distributed and/or modified under the
% conditions of the LaTeX Project Public License, either version 1.3
% of this license or (at your option) any later version.
% The latest version of this license is in
%   http://www.latex-project.org/lppl.txt
% and version 1.3 or later is part of all distributions of LaTeX
% version 2005/12/01 or later.
%
% This work has the LPPL maintenance status `author-maintained'.
% 
% The Current Maintainer and author of this work is Stefan Berthold.
%
% This work consists of all files listed in manifest.txt.
% \fi
%
% \iffalse
%<*driver>
\ProvidesFile{\jobname.dtx}
%</driver>
%<collection|report|masters>\NeedsTeXFormat{LaTeX2e}[1999/12/01]
%<collection>\ProvidesClass{kaucollection}
%<report>\ProvidesClass{kaureport}
%<masters>\ProvidesClass{kaumasters}
%<*collection|report|masters>
    [2012/04/18 v1.8 Karlstad University thesis layout]
%</collection|report|masters>
%
%<*driver>
\documentclass[a4paper]{ltxdoc}
\EnableCrossrefs
\CodelineIndex
\RecordChanges
\begin{document}
  \DocInput{\jobname.dtx}
\end{document}
%</driver>
% \fi
%
% \CheckSum{298}
%
% \CharacterTable
%  {Upper-case    \A\B\C\D\E\F\G\H\I\J\K\L\M\N\O\P\Q\R\S\T\U\V\W\X\Y\Z
%   Lower-case    \a\b\c\d\e\f\g\h\i\j\k\l\m\n\o\p\q\r\s\t\u\v\w\x\y\z
%   Digits        \0\1\2\3\4\5\6\7\8\9
%   Exclamation   \!     Double quote  \"     Hash (number) \#
%   Dollar        \$     Percent       \%     Ampersand     \&
%   Acute accent  \'     Left paren    \(     Right paren   \)
%   Asterisk      \*     Plus          \+     Comma         \,
%   Minus         \-     Point         \.     Solidus       \/
%   Colon         \:     Semicolon     \;     Less than     \<
%   Equals        \=     Greater than  \>     Question mark \?
%   Commercial at \@     Left bracket  \[     Backslash     \\
%   Right bracket \]     Circumflex    \^     Underscore    \_
%   Grave accent  \`     Left brace    \{     Vertical bar  \|
%   Right brace   \}     Tilde         \~}
%
%
% \changes{v1.0}{2011/02/01}{SBe's licentiate version}
% \changes{v1.1}{2011/05/06}{Initial public release}
% \changes{v1.2}{2012/12/14}{Support for babel.sty,\\Documentation added}
% \changes{v1.3}{2012/12/15}{Merged the classes' dtx files}
% \changes{v1.4}{2012/12/20}{Fixed babel.sty support}
% \changes{v1.5}{2012/12/20}{Added templates for kaucollection and kaureport}
% \changes{v1.6}{2012/12/22}{Merged the styles' dtx files}
% \changes{v1.7}{2013/04/12}{Added approval page to kaumasters}
% \changes{v1.8}{2013/04/18}{Improved author's signatures on the approval page of kaumasters}
%
% \GetFileInfo{\jobname.dtx}
%
% \DoNotIndex{\newcommand,\newenvironment}
%
% \StopEventually{\PrintChanges\PrintIndex}
%
% \title{The \textsf{\jobname} package\thanks{This document
%   corresponds to \textsf{\jobname}~\fileversion, dated \filedate.}}
% \author{Stefan Berthold \\ \texttt{stefan.berthold@kau.se}}
%
% \maketitle
%
% \section{Introduction}
%
% The \textsf{\jobname} package provides \LaTeX{} classes for master's theses (\textsf{kaumasters} class), collection theses (\textsf{kaucollection} class), and monographs (\textsf{kaureport} class). The latest version can be obtained from
% \begin{quote}
%   \texttt{http://github.com/ZjMNZHgG5jMXw/kauthesis}\quad.
% \end{quote}
% The user's manual is found in \texttt{kauguide.pdf}. This file provides the source code documentation to all three classes.
%
% \section{Usage}
%
% \subsection{Master's thesis}
%
% This template can be found in \texttt{kaumasterstemplate.tex}.
% The \textsf{kaumasters} class is quite similar to \LaTeX's \textsf{article} class. After loading the class, title, author, and institute have to be declared. Different authors in author lists can be separated by commata (``,''). The formatting will be done by \LaTeX.
%    \begin{macrocode}
%<*masterstemplate>
\documentclass{kaumasters}
\title{Title}
\author{Author}
\supervisor{Super}
\examiner{Exam}
\institute{Department of Mathematics and Computer Science}
\place{Place}
%    \end{macrocode}
% The thesis starts with the front matter |\frontmatter|. It will (re)print the title. The macro |\maketitle| does not exist, since the title will be created by Karlstad University Studies.
%    \begin{macrocode}
\begin{document}
\frontmatter
%    \end{macrocode}
% The two mandatory inhabitants of the front matter are the abstract (|abstract| environment) and the |\tableofcontents|. If you like to write acknowledgements, put it between the abstract and the table of contents.
%    \begin{macrocode}
\begin{abstract}
  Abstract.
  \keywords keywords
\end{abstract}
\approvalpage%
\begin{acknowledgements}
  Thanks.
\end{acknowledgements}
\tableofcontents
%    \end{macrocode}
% The main matter starts with |\mainmatter| for adjusting the page numbering. The content in the main matter is your work. Any macro or environment known from \LaTeX's \textsf{article} class can be used here.
%    \begin{macrocode}
\mainmatter
\section{Introduction}
\end{document}
%</masterstemplate>
%    \end{macrocode}
%
% \subsection{Report}
%
% This template can be found in \texttt{kaureporttemplate.tex}.
% Like the master's thesis class, the report class is derived from the \LaTeX{} \textsf{article} class.
%
%    \begin{macrocode}
%<*reporttemplate>
\documentclass{kaureport}
%    \end{macrocode}
% After loading the class, title, author, and institute have to be declared. Different authors in author lists can be separated by commata (``,''). The formatting will be done by \LaTeX.
%    \begin{macrocode}
\title{Title}
\author{Author}
\institute{Department of Mathematics and Computer Science}
\place{Place}
\begin{document}
%    \end{macrocode}
% The two mandatory inhabitants of the front matter are the abstract (|abstract| environment) and the |\tableofcontents|. If you like to write acknowledgements, put it between the abstract and the table of contents.
%    \begin{macrocode}
\frontmatter
\begin{abstract}
  Brief abstract.
  \keywords keywords
\end{abstract}
\begin{acknowledgements}
  Thanks.
\end{acknowledgements}
\tableofcontents
%    \end{macrocode}
% The main matter starts with |\mainmatter| for adjusting the page numbering. The content in the main matter is your work. Any macro or environment known from \LaTeX's \textsf{article} class can be used here.
%    \begin{macrocode}
\mainmatter
\part{Introduction}
\section{Related work}
\part{Conclusions and Future Work}
\end{document}
%</reporttemplate>
%    \end{macrocode}
%
% \subsection{Collection thesis}
%
% This template can be found in \texttt{kaucollectiontemplate.tex}.
% Like the master's thesis class, the report class is derived from the \LaTeX{} \textsf{article} class.
%
%    \begin{macrocode}
%<*collectiontemplate>
\documentclass{kaucollection}
%    \end{macrocode}
% Page headings and can be created using the \textsf{scrheadings} packages.
%    \begin{macrocode}
\usepackage{scrpage2}
\pagestyle{scrheadings}
\renewcommand*\headfont{\normalfont\small}
\rehead{\thepaper}
\lohead{\thepapertitle}
%    \end{macrocode}
% After loading the class, title, author, and institute have to be declared. Different authors in author lists can be separated by commata (``,''). The formatting will be done by \LaTeX.
%    \begin{macrocode}
\title{Title}
\author{Author}
\institute{Department of Mathematics and Computer Science}
\place{Place}
\begin{document}
%    \end{macrocode}
% The two mandatory inhabitants of the front matter are the abstract (|abstract| environment) and the |\tableofcontents|. If you like to write acknowledgements, put it between the abstract and the table of contents.
%    \begin{macrocode}
\frontmatter
\begin{abstract}
  Brief abstract.
  \keywords keywords
\end{abstract}
\begin{acknowledgements}
  Thanks.
\end{acknowledgements}
\tableofcontents
%    \end{macrocode}
% The table of contents is followed by the list of papers.
%    \begin{macrocode}
\listofpapers
%    \end{macrocode}
% The main matter starts with |\mainmatter| for adjusting the page numbering.
%    \begin{macrocode}
\mainmatter
\section{Introduction}
%    \end{macrocode}
% Create a list of summaries of all appended papers with |\listofsummaries|.
%    \begin{macrocode}
\section{Summary of Appended Papers}
\listofsummaries
\section{Conclusions and Future Work}
%    \end{macrocode}
% Append papers, each in an own |kaupaper| environment.
%    \begin{macrocode}
\begin{kaupaper}[ author=Author
                , title=Paper Title
                , reference=reference text.
                , email=author@kau.se
                , summary=Abstract in the intro.
                , participation=I am the main author.
                , label=paper:shorttitle
                ]
%    \end{macrocode}
% Appended papers may have their own titles and abstracts.
%    \begin{macrocode}
  \maketitle
  \begin{abstract}
    Paper abstract.
  \end{abstract}
  Paper content.
\end{kaupaper}
\end{document}
%</collectiontemplate>
%    \end{macrocode}
%
% \section{Implementation}
%
% \subsection{Class options and dependencies}
%
% Class options are neither parsed nor passed to the underlying \textsf{article} class. All macros are derived from the standard \textsf{article} class.
%    \begin{macrocode}
%<*collection|report|masters>
\ProcessOptions\relax
\LoadClass{article}
%    \end{macrocode}
%
% \noindent \textsf{kauclear} provides page style switches in |\cleardoublepage|,
% \textsf{kaumeta} defines the macros relevant for creating title information, e.\,g., |\institute|.
%    \begin{macrocode}
\RequirePackage{kauclear}
\RequirePackage{kaumeta}
\RequirePackage{kaulist}
%    \end{macrocode}
% \textsf{kaustudies} sets the page dimension to the format required by Karlstad University Studies,
% \textsf{kauenvironments} defines environments such as |researchquestions| and |contributions|,
% \textsf{tikz} allows to place content material with absolute coordinates (required by |\vanityquote|).
%    \begin{macrocode}
%<*collection|report>
\RequirePackage{kaustudies}
\RequirePackage{kauenvironments}
\RequirePackage{tikz}
%</collection|report>
%    \end{macrocode}
% \textsf{kaupaper} defines the macros for appending papers to the collection thesis.
%    \begin{macrocode}
%<*collection>
\RequirePackage{kaupaper}
%</collection>
%    \end{macrocode}
% \textsf{kauparts} defines |\kaupart|, a replacement for |\part|.
%    \begin{macrocode}
%<*report>
\RequirePackage{kauparts}
%</report>
%    \end{macrocode}
% Garamond becomes the default font on A4 page layout.
%    \begin{macrocode}
%<*masters>
\RequirePackage[urw-garamond]{mathdesign}
\RequirePackage[paper=a4,pagesize,twoside=semi]{typearea}
%</masters>
%    \end{macrocode}
%
% \subsection{Language support}
%
% \begin{macro}{\acknowledgementname}
% By default, the |acknowledgements| environment will create an unnumbered section with the title ``Acknowledgements'' (or ``Tacks\"agelser'', if Swedish is chosen as default language). The section headline can be changed to any other language by redefining the macro |\acknowledgementname|.
%    \begin{macrocode}
\AtBeginDocument{%
  \@ifpackageloaded{babel}%
    {\newcommand*\acknowledgementname{%
      \iflanguage{swedish}{Tacks\"agelser}{Acknowledgements}}%
    }%
    {\newcommand*\acknowledgementname{Acknowledgements}}%
}
%    \end{macrocode}
% \end{macro}
%
% \begin{macro}{\keywordname}
% The abstract is followed by the list of keywords. By default, the list is introduced by the word ``Keywords'' (and ``Nyckelord'' if the default language is set to Swedish). The list introduction can be changed to any other language by redefining the macro |\keywordname|.
%    \begin{macrocode}
\AtBeginDocument{%
  \@ifpackageloaded{babel}%
    {\newcommand*\keywordname{%
      \iflanguage{swedish}{Nyckelord}{Keywords}}%
    }%
    {\newcommand*\keywordname{Keywords}}%
}
%    \end{macrocode}
% \end{macro}
%
% \begin{macro}{\introname}
% In collection thesis, the main matter starts automatically with an Introductory Summary. The name of the Introductory Summary (``Inledande sammanfattning'' in Swedish) can be changed to any other language by redefining the macro |\introname|.
%    \begin{macrocode}
%<*collection>
\AtBeginDocument{%
  \@ifpackageloaded{babel}%
    {\newcommand*\introname{%
      \iflanguage{swedish}%
        {Inledande sammanfattning}%
        {Introductory Summary}}%
    }%
    {\newcommand*\introname{Introductory Summary}}%
}
%</collection>
%    \end{macrocode}
% \end{macro}
%
% \medskip
% \noindent A master's thesis needs to be approved by the supervisor and the examiner.
%    \begin{macrocode}
%<*masters>
%    \end{macrocode}
% \begin{macro}{\approvalname}
% This macro contains the word ``approved''.
%    \begin{macrocode}
\AtBeginDocument{%
  \@ifpackageloaded{babel}%
    {\newcommand*\approvalname{%
      \iflanguage{swedish}%
        {Godk\"and}%
        {Approved}}%
    }%
    {\newcommand*\approvalname{Approved}}%
}
%    \end{macrocode}
% \end{macro}
% \begin{macro}{\examinationname}
% This macro contains the name of the examination.
%    \begin{macrocode}
\AtBeginDocument{%
  \@ifpackageloaded{babel}%
    {\newcommand*\examinationname{%
      \iflanguage{swedish}%
        {magisterexamen}%
        {Master's degree}}%
    }%
    {\newcommand*\examinationname{Master's degree}}%
}
%    \end{macrocode}
% \end{macro}
% \begin{macro}{\approvaltext}
% The approval text.
%    \begin{macrocode}
\AtBeginDocument{%
  \@ifpackageloaded{babel}%
    {\newcommand*\approvaltext{%
      \iflanguage{swedish}%
        {Denna uppsats \"ar skriven som en del av det arbete som kr\"avs %
          f\"or att erh\aa lla en \examinationname{} i datavetenskap. Allt %
          material i denna rapport, vilket inte \"ar mitt eget, har blivit %
          tydligt identifierat och inget material \"ar inkluderat som %
          tidigare anv\"ants f\"or erh\aa llande av annan examen.}%
        {This thesis is submitted in partial fulfillment of the %
          requirements for the \examinationname{} in Computer Science. %
          All material in this thesis which is not my own work has been %
          identified and no material is included for which a degree has %
          previously been conferred.}}%
    }%
    {\newcommand*\approvaltext{This thesis is submitted in partial %
      fulfillment of the requirements for the \examinationname{} in %
      Computer Science. All material in this thesis which is not my own %
      work has been identified and no material is included for which a %
      degree has previously been conferred.}}%
}
%    \end{macrocode}
% \end{macro}
% \begin{macro}{\supervisorname}
% This macro contains the word ``supervisor''.
%    \begin{macrocode}
\AtBeginDocument{%
  \@ifpackageloaded{babel}%
    {\newcommand*\supervisorname{%
      \iflanguage{swedish}%
        {Handledare}%
        {Supervisor}}%
    }%
    {\newcommand*\supervisorname{Supervisor}}%
}
%    \end{macrocode}
% \end{macro}
% \begin{macro}{\examinername}
% This macro contains the word ``examiner''.
%    \begin{macrocode}
\AtBeginDocument{%
  \@ifpackageloaded{babel}%
    {\newcommand*\examinername{%
      \iflanguage{swedish}%
        {Examinator}%
        {Examiner}}%
    }%
    {\newcommand*\examinername{Examiner}}%
}
%    \end{macrocode}
% \end{macro}
%    \begin{macrocode}
%</masters>
%    \end{macrocode}
%
% \subsection{Title page}
%
% Karlstad University Studies will create the title page for the thesis, thus, creating an own title page is not necessary.
%
% \begin{macro}{\kauths@maketitle}
% The definition of |\maketitle| in the \textsf{article} class creates a title without creating a separate page. The original definition is preserved in |\kauths@maketitle|.
%    \begin{macrocode}
\let\kauths@maketitle\maketitle
%    \end{macrocode}
% \end{macro}
%
% \begin{macro}{\maketitle}
% |\maketitle| does not serve any purpose. When invoked, it produces a warning, refering the user to the |abstract| environment.
%    \begin{macrocode}
\renewcommand\maketitle{%
  \PackageWarning{maketitle does not serve any purpose in this class.
    The abstract environment reproduces the title instead.}%
}
%    \end{macrocode}
% \end{macro}
%
% \subsection{Front matter}
%
% The front matter starts after the title on page~iii. It contains the abstract of the thesis, followed by an optional acknowledgements section and the table of contents. In collection theses and reports, the keywords are defined in the front matter.
%
% \begin{macro}{\kauths@abstract}
% The |abstract| environment of the \textsf{article} class provides a perfect layout for article abstracts in collection theses. It will be used for the abstracts of appended papers and preserved in the macro |\kauths@abstract|.
%    \begin{macrocode}
\let\kauths@abstract\abstract%
\let\kauths@endabstract\endabstract%
%    \end{macrocode}
% \end{macro}
%
% \begin{macro}{\frontmatter}
% The front matter is started with |\frontmatter|. It is assumed that this macro directly runs after the optional |\maketitle|.
%    \begin{macrocode}
\newcommand*\frontmatter{%
  \clearpage%
%    \end{macrocode}
% The front matter starts on page~iii, i.\,e., two pages are reserved for the title.
%    \begin{macrocode}
  \setcounter{page}{3}%
  \renewcommand\thepage{\roman{page}}%
%    \end{macrocode}
% \begin{environment}{abstract}
% The |abstract| environment in the front matter creates a title and an unnumbered abstract section.
%    \begin{macrocode}
  \renewenvironment{abstract}{%
%    \end{macrocode}
% \begin{macro}{\keywords}
% The abstract's last line in collection theses and reports is supposed to be a list of keywords preceeded by |\keywords|.
%    \begin{macrocode}
    \newcommand\keywords{\paragraph{\keywordname:}}%
%    \end{macrocode}
% \end{macro}
% \begin{environment}{english}
% It is common to append an English translation of the abstract, if the thesis or report is not written in English.
%    \begin{macrocode}
    \newenvironment{english}{%
      \renewcommand\keywords{\paragraph{Keywords:}}%
      \section*{Abstract}%
    }{%
    }
%    \end{macrocode}
% \end{environment}
% The abstract starts by repeating the title, author, and the institue.
%    \begin{macrocode}
    \cleardoublepage%
    \section*{\@title}%
    \textsc{\prettylist{\@author}}\par%
    \noindent\textit{\@institute}%
    \section*{\abstractname}%
  }{%
%    \end{macrocode}
% The thesis continues after the abstract on an odd page.
%    \begin{macrocode}
    \cleardoublepage%
  }%
}
%    \end{macrocode}
% \end{environment}
% \end{macro}
%
% \begin{macro}{\approvalpage}
% The macro |\approvalpage| creates an approval page for Master's theses.
%    \begin{macrocode}
%<*masters>
\newcommand\approvalpage{%
  \cleardoublepage%
  \newcommand\sig{\makebox[7cm]{\hrulefill}\\}%
  \newcommand\signer[1]{\begin{quote}\sig ##1\end{quote}\mbox{}}%
  \noindent\approvaltext\\\mbox{}\bigskip%
  \foreach \x in \@author {\signer{\x}}%
  \\\mbox{}\bigskip\\\mbox{}\bigskip%
  \approvalname,\\\mbox{}\bigskip%
  \begin{quote}%
    \sig\supervisorname: \@supervisor\\\mbox{}\bigskip\\%
    \sig\examinername: \@examiner
  \end{quote}%
}
%</masters>
%    \end{macrocode}
% \end{macro}
%
% \begin{environment}{acknowledgements}
% The environment |acknowledgements| creates an unnumbered section.
%    \begin{macrocode}
\newenvironment{acknowledgements}{%
  \cleardoublepage%
  \section*{\acknowledgementname}%
}{%
%    \end{macrocode}
% The author's name is reused to create an attribution line at the end of the acknowledgements section.
%    \begin{macrocode}
  \par\bigskip\bigskip\bigskip\noindent\@place, \@date\hfill \prettylist{\@author}%
  \cleardoublepage%
}
%    \end{macrocode}
% \end{environment}
%
% \begin{macro}{\kauths@tableofcontents}
% The |\tableofcontents| definition from the \textsf{article} class is preserved. It is not reused in this class.
%    \begin{macrocode}
\let\kauths@tableofcontents\tableofcontents
%    \end{macrocode}
% \end{macro}
%
% \begin{macro}{\tableofcontents}
% The redefined macro makes the table of contents start on an odd page.
%    \begin{macrocode}
\renewcommand\tableofcontents{%
  \cleardoublepage%
  \kauths@tableofcontents%
}
%    \end{macrocode}
% \end{macro}
%
% \subsection{Main matter}
%
% The main matter is started with |\mainmatter|. The macro resets the page counter to page~1 and, for collection theses, it immediately creates the part Introductory Summary.
%
% \begin{macro}{\mainmatter}
% The optional argument of |\mainmatter|\oarg{body material} can be used to get access to the part's title page, e.\,g., for adding a quote with |\vanityquote|.
%    \begin{macrocode}
\newcommand\mainmatter[1][]{%
  \cleardoublepage%
%    \end{macrocode}
% The original definition of the |abstract| environment is restored.
%    \begin{macrocode}
  \renewenvironment{abstract}{\kauths@abstract}{\kauths@endabstract}
%    \end{macrocode}
% The page counter is set to page~1.
%    \begin{macrocode}
  \setcounter{page}{1}%
  \pagenumbering{arabic}%
%    \end{macrocode}
% The Introductory Summary title page is created with |\introname| and \meta{body material}.
%    \begin{macrocode}
%<*collection>
  \kaupart*[body={#1}]{\introname}%
%</collection>
}
%    \end{macrocode}
% \end{macro}
%
% \begin{macro}{\vanityquote}
% The macro |\vanityquote|\marg{quote}\marg{reference} can be used in collection theses and reports to add quotes to the cover pages of parts, e.\,g., in \meta{body material} of |\mainmatter|.
% The first argument, \meta{quote}, is the quote and the second, \meta{reference}, is the reference to the person which is quoted.
%    \begin{macrocode}
%<*collection|report>
\newcommand\vanityquote[2]{%
%    \end{macrocode}
% The quote is placed relative to the borders of the page. TikZ is doing the actual placing.
%    \begin{macrocode}
  \begin{tikzpicture}[remember picture,overlay]%
%    \end{macrocode}
% The placing reference is the lower right corner of the page.
%    \begin{macrocode}
    \path (current page.south east) +(-3,3.5)%
      node%
        [ anchor=south east%
        , font=\large%
        , text width=25em%
%    \end{macrocode}
% The closing quotation marks are placed outside the quote's right margin.
%    \begin{macrocode}
        ] {\raggedleft``#1\rlap{''}%
           \normalfont\par\medskip\itshape #2\par};%
  \end{tikzpicture}%
}
%</collection|report>
%    \end{macrocode}
% \end{macro}
%
% \begin{macro}{\part}
% In the \textsf{kaureport} class, the |\part|\marg{headline} is replaced by |\kaupart|.
%    \begin{macrocode}
%<*report>
\let\part\kaupart
%</report>
%</collection|report|masters>
%    \end{macrocode}
% \end{macro}
%
% \Finale
\endinput
