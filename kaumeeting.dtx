% \iffalse meta-comment
%% kaumeeting.dtx
%% Copyright (c) 2011-2012 Stefan Berthold <stefan.berthold@kau.se>
%
% This file is part of the kauthesis bundle.
%
% This work may be distributed and/or modified under the
% conditions of the LaTeX Project Public License, either version 1.3
% of this license or (at your option) any later version.
% The latest version of this license is in
%   http://www.latex-project.org/lppl.txt
% and version 1.3 or later is part of all distributions of LaTeX
% version 2005/12/01 or later.
%
% This work has the LPPL maintenance status `author-maintained'.
% 
% The Current Maintainer and author of this work is Stefan Berthold.
%
% This work consists of all files listed in manifest.txt.
% \fi
%
% \iffalse
%<*driver>
\ProvidesFile{\jobname.dtx}
%</driver>
%<class>\NeedsTeXFormat{LaTeX2e}[1999/12/01]
%<class>\ProvidesClass{kaumeeting}
%<*class>
    [2011/11/18 v1.0 Karlstad University meeting record]
%</class>
%
%<*driver>
\documentclass[a4paper]{ltxdoc}
\usepackage{url}
\EnableCrossrefs
\CodelineIndex
\RecordChanges
\begin{document}
  \DocInput{\jobname.dtx}
\end{document}
%</driver>
% \fi
%
% \CheckSum{92}
%
% \CharacterTable
%  {Upper-case    \A\B\C\D\E\F\G\H\I\J\K\L\M\N\O\P\Q\R\S\T\U\V\W\X\Y\Z
%   Lower-case    \a\b\c\d\e\f\g\h\i\j\k\l\m\n\o\p\q\r\s\t\u\v\w\x\y\z
%   Digits        \0\1\2\3\4\5\6\7\8\9
%   Exclamation   \!     Double quote  \"     Hash (number) \#
%   Dollar        \$     Percent       \%     Ampersand     \&
%   Acute accent  \'     Left paren    \(     Right paren   \)
%   Asterisk      \*     Plus          \+     Comma         \,
%   Minus         \-     Point         \.     Solidus       \/
%   Colon         \:     Semicolon     \;     Less than     \<
%   Equals        \=     Greater than  \>     Question mark \?
%   Commercial at \@     Left bracket  \[     Backslash     \\
%   Right bracket \]     Circumflex    \^     Underscore    \_
%   Grave accent  \`     Left brace    \{     Vertical bar  \|
%   Right brace   \}     Tilde         \~}
%
%
% \changes{v1.0}{2012/12/14}{Initial public release}
%
% \GetFileInfo{\jobname.dtx}
%
% \DoNotIndex{\newcommand,\newenvironment}
%
% \StopEventually{\PrintChanges\PrintIndex}
%
% \title{The \textsf{\jobname} class\thanks{This document
%   corresponds to \textsf{\jobname}~\fileversion, dated \filedate.}}
% \author{Stefan Berthold \\ \texttt{stefan.berthold@kau.se}}
%
% \maketitle
%
% \section{Introduction}
%
% The \textsf{\jobname} class is part of the kauthesis package. The class provides macros for writing meeting records. The latest version can be obtained from
% \begin{quote}
%   \texttt{http://github.com/ZjMNZHgG5jMXw/kauthesis}\quad.
% \end{quote}
%
% \noindent
% This class requires the {\small URW}~Garamond font. The latest version can be obtained from
% \begin{quote}
%   \texttt{http://www.ctan.org/pkg/urw-garamond}\quad.
% \end{quote}
%
% \noindent
% This class also requires the logo of Karlstad Univerity. The latest version can be obtained from
% \begin{quote}
%   \texttt{http://www.kau.se/om-universitetet/pressinformation/ladda-hem\-logotyp}\quad.
% \end{quote}
%
% \section{Usage}
% \iffalse
%<*template>
% \fi
%
% The following example can be found in \texttt{kaumeetingtemplate.tex}.
%
% \noindent
% After loading the class, complete the meta data with |\metadata|\marg{data}.
%    \begin{macrocode}
\documentclass{kaumeeting}
\metadata%
%    \end{macrocode}
% The subject of the meeting, e.\,g., a course name.
%    \begin{macrocode}
  { meeting={}
%    \end{macrocode}
% The person who is writing the protocol.
%    \begin{macrocode}
  , protocol={}
%    \end{macrocode}
% The place of the meeting.
%    \begin{macrocode}
  , place={}
%    \end{macrocode}
% The calendar date of the meeting.
%    \begin{macrocode}
  , date={}
%    \end{macrocode}
% The time when the meeting began.
%    \begin{macrocode}
  , start={}
%    \end{macrocode}
% The time when the meeting was over.
%    \begin{macrocode}
  , end={}
%    \end{macrocode}
% The persons who attended the meeting.
%    \begin{macrocode}
  , participants={}
%    \end{macrocode}
% Persons excused from the meeting.
%    \begin{macrocode}
  , excused=none
%    \end{macrocode}
% Persons missing from the meeting without excuse.
%    \begin{macrocode}
  , missing={}
%    \end{macrocode}
% The distribution list of the meeting records.
%    \begin{macrocode}
  , distribution={}
%    \end{macrocode}
% The faculty and department names.
%    \begin{macrocode}
  , faculty={Faculty of Health, Science and Technology}%
  , department={Department of Mathematics and Computer Science}%
%    \end{macrocode}
% The name (or location) of Karlstad University's logo.
%    \begin{macrocode}
  , logofile={kau-logo-tryck}
  }
%    \end{macrocode}
% The macro |\maketitle| creates the record header from the meta data.
%    \begin{macrocode}
\begin{document}
\maketitle
%    \end{macrocode}
% After |\maketitle|, content can be added like in a \LaTeX{} \textsf{article}.
%    \begin{macrocode}
\end{document}
%    \end{macrocode}
% \iffalse
%</template>
% \fi
%
% \section{Implementation}
% \iffalse
%<*class>
% \fi\setcounter{CodelineNo}{0}
%
% \subsection{Class options and dependencies}
%
% Class options are neither parsed nor passed to the underlying \textsf{article} class.
%    \begin{macrocode}
\ProcessOptions\relax
%    \end{macrocode}
%
% \noindent \textsf{\jobname} derives all macros from the standard \textsf{article} class.
%    \begin{macrocode}
\LoadClass{article}
%    \end{macrocode}
%
% \noindent Garamond is chosen as default font.
%    \begin{macrocode}
\RequirePackage[urw-garamond]{mathdesign}
%    \end{macrocode}
% \noindent A4 is set as the default page layout.
%    \begin{macrocode}
\RequirePackage[paper=a4,pagesize]{typearea}
%    \end{macrocode}
%    \begin{macrocode}
\RequirePackage{xkeyval}
\RequirePackage{graphicx}
\RequirePackage{tikz}
\RequirePackage{tabularx}
\RequirePackage{ragged2e}
\RequirePackage{xifthen}
%    \end{macrocode}
%
% \subsection{Meta data}
%
% \begin{macro}{\metadata}
%    \begin{macrocode}
\define@cmdkeys[kaumet]{general}[kaumet@]%
  { meeting, protocol, place%
  , date, start, end%
  , participants, excused, missing%
  , distribution%
  , faculty, department%
  , logofile%
  }
\newcommand\metadata[1]{\setkeys[kaumet]{general}{#1}}
\setkeys[kaumet]{general}%
  { meeting=, protocol=, place=%
  , date=\today, start=, end=%
  , participants=, excused=, missing=%
  , distribution=%
  , faculty={Faculty of Health, Science and Technology}%
  , department={Department of Mathematics and Computer Science}%
  , logofile={kau-logo-tryck}
  }
%    \end{macrocode}
% \end{macro}
%
% \subsection{Title page}
%
% \begin{macro}{\kaumet@maketitle}
% The definition of |\maketitle| in the \textsf{article} class creates a title without creating a separate page. \textsf{\jobname} redefines the macro and preserves the original definition in |\kaumas@maketitle|.
%    \begin{macrocode}
\let\kaumet@maketitle\maketitle
%    \end{macrocode}
% \end{macro}
%
% \begin{macro}{\maketitle}
%    \begin{macrocode}
\renewcommand\maketitle{
  \thispagestyle{empty}
  \begin{tikzpicture}[overlay,remember picture]
    % top
    \path (current page.north)
      node[below=5mm, text centered, text width=\textwidth] (top)
      {\includegraphics[width=33mm]{\kaumet@logofile}\\[5mm]%
      \kaumet@faculty\\%
      \kaumet@department}
    ;
  \end{tikzpicture}
  \vspace{1.5cm}\\
  \mbox{}\hfill\kaumet@date\\
  \centering \Large\scshape Meeting Record\\
  \justifying\normalfont\bigskip\small\noindent
  \begin{tabularx}{\textwidth}{@{}>{\bfseries\vphantom{Xy}}rX}
    subject & \kaumet@meeting\\
    place & \kaumet@place\\
    start & \kaumet@start\\
    end & \kaumet@end\\
    participants & \kaumet@participants\\
    recording & \kaumet@protocol\\
    excused & \kaumet@excused\\
    missing & \kaumet@missing\\
    distribution & \kaumet@distribution
  \end{tabularx}\smallskip
  \justifying\normalfont\normalsize
}
%    \end{macrocode}
% \end{macro}
% \iffalse
%</class>
% \fi
%
% \Finale
\endinput

